%\section{Discussion}
Our results partially support our hypothesis, as we were correct to assume that increasing the pullback distance would increase the range, however, the trend we found was quadratic rather than linear. The assumption was made that the elastic was a linear spring, which follows Hooke's Law - the force exerted by a spring as it returns to its original position is directly proportional to the distance of deformation. However, substituting Hooke’s Law into the equation 
$F=1/2mv^2$ reveals that a linear increase in the force applied to an object results in only a square root increase in the velocity (and therefore range) of the projectile. Therefore, we should expect a square root trend in the range of the projectile, however, the data shows a quadratic increase in the range. This could be due to not pushing the elastic to its extremes, either too low or too high, which would present as skewed data in our experimental observations. In order to test if there is actually a square root increase, a stronger elastic should be used, in order to test higher forces. Because of the $v^2$ term, the trend would appear linear at low forces, and only become noticeably curved at higher forces. Using a stronger elastic would allow testing of higher forces while still ensuring that the spring is behaving linearly. \\

One potential source of error in the experiment is the human error involved in firing the projectile. Sometimes, the projectile would be fired at an angle or the elastic would get twisted, putting spin on the projectile, both of which would reduce the measured range. To account for this, as many trails as possible were completed and outliers were removed where there were serious problems when the projectile was fired. That being said, it would be beneficial to automate the launch mechanism by connecting it to a switch or button that would be pressed to release the elastic. This would make the launcher more precise and remove the systematic error.
Furthermore, the materials used in the experiment were of lower quality, and were not particularly sturdy. This caused the launcher to shift around as projectiles were being fired, which would introduce random error into the experiment. With more time and a larger budget, better materials could be purchased and assembled to improve the quality of the launcher and the reliability of the data gathered.

Overall, our data modelled a strong correlative trend between Distance Travelled and Pullback Distance as shown in Figure \ref{fig:graph1}, with an $R^2$ value of 0.992. However, we expected the elastic to behave linearly, indicating that there were errors present in either experimental design, or some other extraneous error that could have skewed our results. Even with the slightly skewed data, we can still conclude that when pullback distance of a catapult is increased, the range will increase at an almost-linear rate. 

