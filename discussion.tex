\subsection{Methodology}
One unique part of our methodology that proved crucial to the statistical
significance and accuracy of our data was the addition of a 250g weight on top of the
Smart Cart. Adding the various cardboard attachments used to modify the aerodynamic properties of the
Smart Cart would increase the total weight of our modified Smart Cart system. It would be possible to
account for the variance in weight by measuring each cardboard attachment along with the tape used to
affix it; however, a lack of time meant that it would be more efficient to make the additional weight
negligible by increasing the overall weight of the Smart Cart itself. Although this may introduce some
variance into our results, the relative effect of the additional weight is acceptable given how little
the attachments weighed relative to the weighted Smart Cart.

\subsection{Air Resistance}
A major goal of this experiment was to determine whether or not air resistance actually existed, or if 
it had enough of an effect to affect our data in a manner that was statistically significant. If air resistance
were to exist and be statistically significant with respect to the data collected, then there should be a statistically significant
decrease in acceleration due to the increasing drag forces. After collecting multiple sets of
data and removing outliers based on inter-quartile ranges, the data for each trial were individually plotted on a graph.
Specifically, when acceleration relative to time was plotted for each trial, as well as a linear regression for acceleration relative to time,
it becomes clear that when cross-sectional area is increased, there is a time-dependent, increasing drag force that decreases acceleration.
Referring to \ref{chart:cardboard12x20.75}, \ref{chart:cardboard15.78x15.78}, \ref{chart:equilateralwedge}, \ref{chart:orange}, \ref{chart:orangepocket}, \ref{chart:sharpwedge}, and \ref{chart:sharpwedgeonorange},
each of their respective linear regressions of acceleration all have negative slopes and decrease with respect to time - signifying an overall deceleration.
Furthermore, when referring to the base case, \ref{chart:nothing}, it can be seen that the slope of the linear regression of acceleration is
practically zero. The base case did not have a cardboard attachment nor an increased cross-sectional area.
This shows that increasing cross-sectional area will affect the aerodynamic properties of the cart,
which will create drag forces that decelerate the cart.

\subsection{Calculating the Drag Coefficient (\textit{K})}