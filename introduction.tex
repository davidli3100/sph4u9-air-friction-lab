%\section{Introduction}

Acids and bases are substances that have possessed multiple definitions throughout history. It has only been in relatively recent times where a definitive definition has been developed. Originally, acids were commonly known for being sour, electrically conductive, and corrosive \cite{story_bench--bedside_2004}. Throughout the majority of history, acids were characterized by their sourness, and it was one of the primary methods for detecting an acid - in fact, the word acid is derived from the Latin word \textit{acidus}, meaning sour \cite{story_bench--bedside_2004}. These classifications of acids and bases were first created by Svante Arrhenius in the 1880s. Arrhenius defined acids to be a substance that dissociates in water to produce H+ ions, and bases to be substances that dissociate in water to produce OH- ions \cite{noauthor_nobel_nodate}. These definitions represented a significant advancement in acid-base theory, however, they could not account for all bases, or reactions that occured outside of water. Shortly after World War I, Bronsted and Lowry developed another definition for acids and bases that filled in the gaps of Arrhenius' definitions: they defined acids to be H+ donators, and bases to be H+ acceptors \cite{story_bench--bedside_2004}.

In this experiment, the Bronsted Lowry definition of acids and bases can be seen in this balanced neutralization reaction between acetic acid and sodium hydroxide:
\[CH_{3}COOH + NaOH \rightarrow H_{2}O + NaCH_{3}COOH\]

Generally, as can be seen in this example, neutralizing an acid with a base produces water and a salt. Even though acetic acid has four hydrogens, it is still a weak acid since only one H+ proton is ionized and donated to OH- to make water. This means only one NaOH molecule would be required to neutralize a molecule of acetic acid. Based on the strength of acetic acid, it is then possible to determine the concentration of acetic acid in vinegar by using a process called titration. 

In a titration, acids and bases are added together in certain quantities in order to fully neutralize the solution to find the concentration of a certain substance \cite{noauthor_acid-base_2013}. The strength of acids and bases can be measured using the logarithmic pH scale, which effectively just measures H+ proton concentration of a particular solution. At a pH of 7, H+ proton concentrations are equivalent to that of OH- concentration, or in other words, the solution is completely neutral in terms of acidity. When pH is greater than 7, then the solution is basic, which means that there are less H+ than OH-. Finally, a pH that is less than 7 represents an acidic solution with more H+ than OH-. 

In this experiment, a solution with an unknown concentration of acetic acid was titrated over several trials. NaOH was added to a burette, and slowly added to the acetic acid solution to neutralize the acid and bring the solution to a neutral pH of 7 (the equivalence point). Neutralizations are not discernible to the naked eye, thus, indicators must be added to the solution to reflect the changing pH as a titration is performed. In this experiment, phenolphthalein was added to the acetic acid solution to accurately determine when the equivalence point was reached via titration. Phenolphthalein is an indicator that remains colourless in an acidic solution, and red in a basic solution - starting at a pH of 8.0 \cite{noauthor_phenolphthalein_nodate}. With the phenolphthalein indicator, reaching a point where the solution became pink for 10+ seconds before becoming colourless once again meant that the pH of the solution was ~7, meaning that the equivalence point had been reached. 

Using the moles of acid and base obtained from titrating acetic acid with sodium hydroxide, the concentration of acetic acid in vinegar can then be determined through various stoichiometric calculations. 

