%\section{Methods}
\begin{enumerate}
    \item The wedge was placed underneath the middle of the cardboard so that it made an angle of 43.2 degrees with the horizontal.
    \item The 2 rods were securely taped vertically, 7cm away from the middle of the cardboard. 
    \item The 30cm ruler was taped off to the side of the middle of the cardboard to measure the pullback distance.
    \item The elastic was placed around the 2 rods so that the middle of the elastic lined up with the middle of the cardboard.
    \item A meter stick was placed on the ground where the projectile will land to measure its range. 
        \begin{itemize}
            \item If the range was longer than a meter, the beginning of the meter stick was placed at a fixed offset from the catapult.
        \end{itemize}
    \item Both sides of the elastic were pulled back to the required distance and the projectile was placed in the middle of the elastic.
    \item The projectile was released and its movement was recorded with a camera. The video was analyzed to determine the final range.
    \item Steps 6-7 were repeated 6 more times to minimize experimental error.
    \item Steps 5-8 were repeated with the other pullback distances until there was enough data for all the distances chosen.
\end{enumerate}
